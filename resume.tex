\documentclass[lualatex, article]{jlreq}

% Packages
\usepackage{luatexja}     % Japanize
\usepackage[backend=biber,
            style=alphabetic,
            sorting=none,
            ]{biblatex}    % bibliography
\usepackage{amsthm}        % proof environment
\usepackage{algorithmic}   % algorithmic environment
\usepackage{algorithm}     % algorithm environment

\addbibresource{references.bib}

% Japanize
\newtheorem{theorem}{定理}
\newtheorem{lemma}{補題}
\newtheorem{definition}{定義}
\newtheorem{corollary}{系}
\renewcommand{\proofname}{\textbf{証明}}

% Metadata
\title{スケルトンの役割と応用}
\author{岡山大輝}
\date{\today}

\begin{document}

\maketitle{}

\begin{abstract}
  スケルトンは、骨格や基盤を指す用語であり、多くの分野で重要な役割を果たしている。本稿では、スケルトンの基本的な概念とその利用価値について述べる。特に、近年注目されているスケルトンの応用例をいくつか紹介し、その有用性を検討する。
\end{abstract}

\section{はじめに}

スケルトンという用語は、多くの異なる分野で使用される。生物学では、スケルトンは骨格を指し、身体の構造を支える役割を果たす。建築学では、建物の骨組みを指し、全体の形状と強度を保つ役割を担う。また、コンピュータサイエンスでは、データ構造やアルゴリズムの基盤を指すことがある。本論文では、これらの異なるスケルトンの概念とその応用について詳述する。

\section{生物学におけるスケルトン}

生物学におけるスケルトンは、主に骨や軟骨で構成されており、身体の支持、運動、保護の機能を果たす。人間の骨格は206個の骨から成り、これらが協調して身体の構造を支える。スケルトンの健康は、栄養素やホルモンバランス、運動などに大きく依存する。近年の研究では、スケルトンの再生医療や骨粗鬆症の治療法の開発が進んでいる~\cite{okayama2024how}。

\section{建築学におけるスケルトン}

建築におけるスケルトンは、建物の骨組みやフレームを指し、全体の構造を支持する役割を果たす。現代の建築では、鋼鉄やコンクリートなどの素材が使用され、高層ビルや橋梁の建設において重要な要素となっている。スケルトン構造は、建物の耐震性や耐久性を向上させるために設計されることが多い。

\section{コンピュータサイエンスにおけるスケルトン}

コンピュータサイエンスの分野では、スケルトンはアルゴリズムやデータ構造の基本的なフレームワークを指す。例えば、ツリー構造やグラフ構造は、データの効率的な管理と検索に用いられる。また、スケルトンアルゴリズムは、並列処理や分散システムの設計において重要な役割を果たす。これにより、大規模なデータセットの処理が可能となり、ビッグデータ解析や機械学習の分野での応用が進んでいる。

\section{スケルトンの応用例}

近年、スケルトンの応用範囲はさらに広がっている。例えば、ロボティクスでは、ロボットのフレームワークとしてスケルトンが使用され、柔軟な動作と耐久性が求められる。また、バイオメカニクスでは、人間の動作解析やリハビリテーションのための支援技術として、スケルトンモデルが利用されている。さらに、仮想現実や拡張現実の分野でも、スケルトンを基にしたモデリングが重要な役割を果たしている。

\section{結論}

本論文では、スケルトンの基本的な概念とその利用価値について述べた。生物学、建築学、コンピュータサイエンスの各分野において、スケルトンは重要な役割を果たしている。また、これらの分野でのスケルトンの応用例を通じて、その有用性が再認識されている。今後もスケルトンの研究と応用が進展し、さらなる発展が期待される。

\printbibliography[title=引用文献]{}


\end{document}
