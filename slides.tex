\documentclass[aspectratio=169]{beamer}

% Packages
\usepackage{luatexja}

% \usepackage[backend=biber, style=numeric]{biblatex}
% TODO: My biber (2.20) and biblatex (3.19) versions are incompatible, but I am tired to upgrade/downgrade them.
% Instead, I am temporarily using bibtex as a backend.
% See detail https://github.com/orgs/Homebrew/discussions/3003#discussioncomment-9153790
\usepackage[backend=bibtex, style=alphabetic, sorting=none]{biblatex}
\addbibresource{references.bib}

% Theme
\usetheme{metropolis}
\usecolortheme{default}

% Fonts
\setbeamerfont{title}{size=\large,series=\bfseries}
\usefonttheme{professionalfonts}
\usefonttheme{structurebold}
\setbeamerfont{title}{size=\LARGE}
\setbeamerfont{date}{size=\small}
\setbeamerfont{frametitle}{size=\large,series=\bfseries}
\usepackage[T1]{fontenc}
\usepackage{mlmodern}
\usepackage{luatexja-otf}
\renewcommand{\kanjifamilydefault}{\gtdefault}
\renewcommand{\familydefault}{\sfdefault}

% Beamer settings
\title{スケルトンの役割と応用}
\author{岡山大輝}
\institute{兵庫県立大学}
\date{\today}

\AtBeginSection[]
{
  \begin{frame}{目次}
    \tableofcontents[currentsection]
  \end{frame}
}

\begin{document}

\begin{frame}
  \titlepage{}
\end{frame}

\begin{frame}{目次}
  \tableofcontents
\end{frame}

\section{概要}

\begin{frame}{概要}
  \begin{itemize}
    \item スケルトンは、骨格や基盤を指す用語であり、多くの分野で重要な役割を果たす。
    \item 本スライドでは、生物学、建築学、コンピュータサイエンスにおけるスケルトンの利用価値と応用例を紹介する。
  \end{itemize}
\end{frame}

\section{生物学におけるスケルトン}

\begin{frame}{生物学におけるスケルトン}
  \begin{itemize}
    \item 骨や軟骨で構成され、身体の支持、運動、保護の機能を果たす。
    \item 人間の骨格は206個の骨から成る。
    \item 近年の研究: スケルトンの再生医療、骨粗鬆症の治療法~\cite{okayama2024how}。
  \end{itemize}
\end{frame}

\section{建築学におけるスケルトン}

\begin{frame}{建築学におけるスケルトン}
  \begin{itemize}
    \item 建物の骨組みやフレームを指し、全体の構造を支持する役割。
    \item 鋼鉄やコンクリートなどの素材を使用。
    \item 耐震性や耐久性の向上に貢献。
  \end{itemize}
\end{frame}

\section{コンピュータサイエンスにおけるスケルトン}

\begin{frame}{コンピュータサイエンスにおけるスケルトン}
  \begin{itemize}
    \item アルゴリズムやデータ構造の基本的なフレームワーク。
    \item ツリー構造やグラフ構造: データの効率的な管理と検索。
    \item 並列処理や分散システムの設計における重要性。
  \end{itemize}
\end{frame}

\section{スケルトンの応用例}

\begin{frame}{スケルトンの応用例}
  \begin{itemize}
    \item ロボティクス: 柔軟な動作と耐久性を持つロボットのフレームワーク。
    \item バイオメカニクス: 人間の動作解析やリハビリテーション支援技術。
    \item 仮想現実や拡張現実: モデリング技術におけるスケルトンの重要性。
  \end{itemize}
\end{frame}

\section{結論}

\begin{frame}{結論}
  \begin{itemize}
    \item スケルトンは多くの分野で重要な役割を果たしている。
    \item その応用範囲は広く、今後も研究と応用が進展すると期待される。
  \end{itemize}
\end{frame}

\begin{frame}{引用文献}
  \printbibliography[title=引用文献]{}
\end{frame}

\end{document}
